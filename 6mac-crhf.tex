% presentation
\documentclass{beamer}
\usetheme[height=7mm]{Rochester}
\usecolortheme{rose}

% handout

%\documentclass[handout]{beamer}
%\usepackage{pgfpages} \pgfpagesuselayout{8 on 1}[a4paper]

%\documentclass[mathserif]{article}
%\usepackage{beamerarticle}

\usepackage{amsmath}
\usepackage{comment}
\usepackage{amssymb,amsfonts}
\usepackage[T1]{fontenc}
\usepackage{lmodern}
\usepackage{tikz}
\usepackage{simpsons}
\usepackage{marvosym}
\usepackage{color}
\usepackage{multirow}
\usepackage{pgffor}
\usepackage[slide,algoruled,titlenumbered,vlined,noend,linesnumbered,]{algorithm2e}

\usefonttheme{structurebold}

\setbeamertemplate{footline}[frame number]
\setbeamertemplate{navigation symbols}{}
\setbeamerfont{smallverb}{size*={73}}
\usefonttheme[onlymath]{serif}
\setbeamertemplate{theorems}[numbered]
\newtheorem{construction}[theorem]{Construction}
\newtheorem{proposition}[theorem]{Proposition}

\AtBeginSection[] {
  \begin{frame}
    \frametitle{Content}
    \tableofcontents[currentsection]
  \end{frame}
  \addtocounter{framenumber}{-1}
}

\usetikzlibrary[shapes.arrows]
\usetikzlibrary{shapes.geometric}
\usetikzlibrary{backgrounds}
\usetikzlibrary{positioning}
\usetikzlibrary{calc}
\usetikzlibrary{intersections}
\usetikzlibrary{fadings}
\usetikzlibrary{decorations.footprints}
\usetikzlibrary{patterns}
\usetikzlibrary{shapes.callouts}
\usetikzlibrary{fit}
%handout

\providecommand{\abs}[1]{\lvert#1\rvert}

\tikzset{every picture/.style={line width=1pt,show background rectangle},background rectangle/.style={fill=blue!10,rounded corners=2ex}}

\author{Yu Zhang}
\institute{Harbin Institute of Technology}
\date[Crypt'20A]{Cryptography, Autumn, 2020}

%% presentation
\documentclass{beamer}
\usetheme[height=7mm]{Rochester}
\usecolortheme{rose}

% handout

%\documentclass[handout]{beamer}
%\usepackage{pgfpages} \pgfpagesuselayout{8 on 1}[a4paper]

%\documentclass[mathserif]{article}
%\usepackage{beamerarticle}

\usepackage{amsmath}
\usepackage{comment}
\usepackage{amssymb,amsfonts}
\usepackage[T1]{fontenc}
\usepackage{lmodern}
\usepackage{tikz}
\usepackage{simpsons}
\usepackage{marvosym}
\usepackage{color}
\usepackage{multirow}
\usepackage{pgffor}
\usepackage[slide,algoruled,titlenumbered,vlined,noend,linesnumbered,]{algorithm2e}

\usefonttheme{structurebold}

\setbeamertemplate{footline}[frame number]
\setbeamertemplate{navigation symbols}{}
\setbeamerfont{smallverb}{size*={73}}
\usefonttheme[onlymath]{serif}
\setbeamertemplate{theorems}[numbered]
\newtheorem{construction}[theorem]{Construction}
\newtheorem{proposition}[theorem]{Proposition}

\AtBeginSection[] {
  \begin{frame}
    \frametitle{Content}
    \tableofcontents[currentsection]
  \end{frame}
  \addtocounter{framenumber}{-1}
}

\usetikzlibrary[shapes.arrows]
\usetikzlibrary{shapes.geometric}
\usetikzlibrary{backgrounds}
\usetikzlibrary{positioning}
\usetikzlibrary{calc}
\usetikzlibrary{intersections}
\usetikzlibrary{fadings}
\usetikzlibrary{decorations.footprints}
\usetikzlibrary{patterns}
\usetikzlibrary{shapes.callouts}
\usetikzlibrary{fit}
%handout

\providecommand{\abs}[1]{\lvert#1\rvert}

\tikzset{every picture/.style={line width=1pt,show background rectangle},background rectangle/.style={fill=blue!10,rounded corners=2ex}}

\author{Yu Zhang}
\institute{Harbin Institute of Technology}
\date[Crypt'20A]{Cryptography, Autumn, 2020}

%\input{1introduction.tex}
%\input{2perfectlysecret.tex}
%\input{3privatekey.tex}


\title{Introduction}

\begin{document}
\maketitle
\begin{frame}
\frametitle{Outline}
\tableofcontents
\end{frame}
\section{Cryptography and Modern Cryptography}
\begin{frame}\frametitle{What is Cryptography?}
\begin{itemize}
\item \textbf{Cryptography}: from Greek \emph{krypt\'os}, ``hidden, secret''; and \emph{gr\'{a}phin}, ``writing''
\item \textbf{Cryptography}: the art of writing or solving codes.\\ (Concise oxford dictionary 2006)
\item \textbf{Codes}: a system of prearranged signals, especially used to ensure secrecy in transmitting messages. \\ (\emph{code word} in cryptography)
\item \textbf{1980s}: from Classic to Modern; from Military to Everyone
\item \textbf{Modern cryptography}: the scientific study of mathematical techniques for securing digital information, systems, and distributed computations against adversarial attacks
\end{itemize}
\end{frame}
\section{The Setting of Private-Key Encryption}
\begin{frame}\frametitle{Private-Key Encryption}
\begin{itemize}
\item \textbf{Goal}: to construct \textbf{ciphers} (encryption schemes) for providing secret communication between two parties sharing \textbf{private-key} (the symmetric-key) in advance
\item \textbf{Implicit assumption}: there is some way of initially sharing a key in a secret manner
\item \textbf{Disk encryption}: the same user at different points in time
\end{itemize}
\end{frame}
\begin{frame}\frametitle{The Syntax of Encryption}
\begin{figure}
\begin{center}
\input{tikz/private-key}
\end{center}
\end{figure}
\begin{itemize}
\item key $k \in \mathcal{K}$, plaintext (or message) $m \in \mathcal{M}$, ciphertext $c \in \mathcal{C}$
\item \textbf{Key-generation} algorithm~$k \gets \mathsf{Gen}$
\item \textbf{Encryption} algorithm~$c:= \mathsf{Enc}_k(m)$
\item \textbf{Decryption} algorithm~$m:= \mathsf{Dec}_k(c)$
\item \textbf{Encryption scheme}: $\Pi = (\mathsf{Gen}, \mathsf{Enc}, \mathsf{Dec})$
\item \textbf{Basic correctness requirement}: $\mathsf{Dec}_k(\mathsf{Enc}_k(m)) = m$
\end{itemize}
\end{frame}
\begin{frame}\frametitle{Securing Key vs Obscuring Algorithm}
\begin{itemize}
\item Easier to maintain secrecy of a short key
\item In case the key is exposed, easier for the honest parties to change the key
\item In case many pairs of people, easier to use the same algorithm, but different keys
\end{itemize}
\begin{alertblock}{Kerckhoffs's principle}
\begin{quote}
The cipher method must not be required to be secret, and it must be able to fall into the hands of the enemy without inconvenience.
\end{quote}	
\end{alertblock}
\end{frame}
\begin{frame}\frametitle{Why ``Open Cryptographic Design''}
\begin{itemize}
\item Published designs undergo public scrutiny are to be stronger
\item Better for security flaws to be revealed by ``ethical hackers''
\item Reverse engineering of the code (or leakage by industrial espionage) poses a serious threat to security
\item Enable the establishment of standards.
\end{itemize}
\end{frame}
\begin{frame}\frametitle{Attack Scenarios}	
\begin{itemize}
\item \textbf{Ciphertext-only}: the adversary just observes ciphertext
\item \textbf{Known-plaintext}: the adversary learns pairs of plaintexts/ciphertexts under the same key
\item \textbf{Chosen-plaintext}: the adversary has the ability to obtain the encryption of plaintexts of its choice
\item \textbf{Chosen-ciphertext}: the adversary has the ability to obtain the decryption of \textbf{other} ciphertexts of its choice
\item \textbf{Passive attack}: COA KPA
\begin{itemize}
\item because not all ciphertext are confidential
\end{itemize}
\item \textbf{Active attack}: CPA CCA
\begin{itemize}
\item when to encrypt/decrypt whatever an adversary wishes?
\end{itemize}
\end{itemize}	
\end{frame}
\section{Historical Ciphers and Their Cryptanalysis}
\begin{comment}
	\begin{frame}\frametitle{Why We Learn Broken Ciphers?}
	\begin{itemize}
	\item To understand the weaknesses of an ``ad-hoc'' approach
	\item To learn that ``simple'' approaches are unlikely to succeed
	\item To feel that ``we are smart enough to do some crypt-analyzing''
	\end{itemize}
	\end{frame}
\end{comment}

\begin{frame}[fragile]\frametitle{Caesar's Cipher}
\begin{quote}
If he had anything confidential to say, he wrote it in cipher, that is, by so changing the order of the letters of the alphabet, that not a word could be made out. If anyone wishes to \alert{decipher} these, and get at their meaning, he must \alert{substitute the fourth letter of the alphabet, namely D, for A}, and so with the others

\rightline{--Suetonius,``Life of Julius Caesar''}
\end{quote}
\begin{itemize}
	\item $\mathsf{Enc}(m)=m+3\mod 26$ \footnote{In fact the quote indicates that decryption involved rotating letters of the alphabet forward 3 positions, $\mathsf{Dec}(c)=c+3\mod 26$}
	\item \textbf{Weakness}: ? %\alert{What is the key?}
\end{itemize}
\begin{exampleblock}{Example}
\verb|begintheattacknow|
%\verb|EHJLQWKHDWWDFNQRZ|
\end{exampleblock}
\end{frame}
\begin{frame}[fragile]\frametitle{Shift Cipher}
\begin{itemize}
\item $\mathsf{Enc}_k(m)=m+k\mod 26$
\item $\mathsf{Dec}_k(c)=c-k\mod 26$
\item \textbf{Weakness}: ? %Fragile under \textbf{Brute-force attack} (exhaustive search)
\end{itemize}
\begin{exampleblock}{Example: Decipher the string}	
\verb|EHJLQWKHDWWDFNQRZ|
\end{exampleblock}
\begin{alertblock}{Sufficient Key Space Principle}
Any secure encryption scheme must have a key space that is not vulnerable to exhaustive search.\footnote{If the plaintext space is larger than the key space.}
\end{alertblock}
\end{frame}
\begin{frame}\frametitle{Index of Coincidence (IC) Method (to find $k$)}
\textbf{Index of Coincidence (IC)}: the probability that two randomly selected letters (pick-then-return) will be identical.

Let $p_i$ denote the probability of $i$th letter in English text.
\[I \overset{\text{def}}{=}\sum_{i=0}^{25} p_i^2 \]
\begin{exampleblock}{Example}
What's the IC of `apple'?
\end{exampleblock}

For a long English text, the IC is $\approx 0.065$.
For $j = 0, 1, \dotsc , 25$, $q_j$ is the probability of $j$th letter in the ciphertext.
\[I_j \overset{\text{def}}{=}\sum_{i=0}^{25} p_i \cdot q_{i+j}\]
\alert{Q: For shift cipher, if $j = k$, then $I_j \approx$ ?}
\end{frame}

\begin{frame}[fragile]\frametitle{Mono-Alphabetic Substitution}
\begin{itemize}
\item \textbf{Idea}: To map each character to a different one in an arbitrary manner
\item \textbf{Strength}: Key space is large $\approx 2^{88}$. \alert{Q: how to count?}
\item \textbf{Weakness}: ? %The mapping of each letter is fixed
\end{itemize}
\begin{exampleblock}{Example}
\verb|abcdefghijklmnopqrstuvwxyz|\\
\verb|XEUADNBKVMROCQFSYHWGLZIJPT|

Plaintext: \verb|tellhimaboutme|\\
Ciphertext: \verb|??????????????|
\end{exampleblock}
\end{frame}
\begin{frame}[fragile]\frametitle{Attack with Statistical Patterns}
\begin{enumerate}
\item Tabulate the frequency of letters in the ciphertext
\item Compare it to those in English text
\item Guess the most frequent letter corresponds to \verb|e|, and so on
\item Choose the plaintext that does ``make sense'' (Not trivial)
\end{enumerate}
\begin{table}
\begin{center}
\caption{Average letter frequencies for English-language text}
\begin{tabular}{|cc|cc|cc|cc|cc|} \hline
e & 12.7\% & t & 9.1\% & a & 8.2\% & o & 7.5\% & i & 7.0\%\\
n & 6.7\% & \_ & 6.4\% & s & 6.3\% & h & 6.1\% & r & 6.0\%\\
d & 4.3\% & l & 4.0\% & c & 2.8\% & u & 2.8\% & m & 2.4\%\\
w & 2.4\% & f & 2.2\% & g & 2.0\% & y & 2.0\% & p & 1.9\%\\
b & 1.5\% & v & 1.0\% & k & 0.8\% & j & 0.2\% & x & 0.2\%\\
q & 0.1\% & z & 0.1\% & & & & & &\\ \hline
\end{tabular}
\end{center}
\end{table}
\end{frame}
\begin{frame}[fragile]\frametitle{Example of Frequency Analysis (Ciphertext)}
\begin{verbatim}
LIVITCSWPIYVEWHEVSRIQMXLEYVEOIEWHRXEXIPFEMVEWHKVS
TYLXZIXLIKIIXPIJVSZEYPERRGERIMWQLMGLMXQERIWGPSRIH
MXQEREKIETXMJTPRGEVEKEITREWHEXXLEXXMZITWAWSQWXSWE
XTVEPMRXRSJGSTVRIEYVIEXCVMUIMWERGMIWXMJMGCSMWXSJO
MIQXLIVIQIVIXQSVSTWHKPEGARCSXRWIEVSWIIBXVIZMXFSJX
LIKEGAEWHEPSWYSWIWIEVXLISXLIVXLIRGEPIRQIVIIBGIIHM
WYPFLEVHEWHYPSRRFQMXLEPPXLIECCIEVEWGISJKTVWMRLIHY
SPHXLIQIMYLXSJXLIMWRIGXQEROIVFVIZEVAEKPIEWHXEAMWY
EPPXLMWYRMWXSGSWRMHIVEXMSWMGSTPHLEVHPFKPEZINTCMXI
VJSVLMRSCMWMSWVIRCIGXMWYMX
\end{verbatim}
\end{frame}
\begin{frame}[fragile]\frametitle{Example of Frequency Analysis (Analysis)}
Count and Guess, Trial and Error.
\begin{table}
\begin{center}
\caption{Analysis Steps}
\begin{tabular}{|r|l|} \hline
Ciphertext & Plaintext \\ \hline
\alert{I}   & \alert{e} \\
\alert{XLI} & \alert{the} \\
\alert{E} & \alert{a} \\
\alert{R}tate & \alert{s}tate \\
atthatt\alert{MZ}e & atthatt\alert{im}e \\
he\alert{V}e & he\alert{r}e \\
remar\alert{A} & remar\alert{k} \\ \hline
\end{tabular}
\end{center}
\end{table}
\end{frame}
\begin{frame}[fragile]\frametitle{Example of Frequency Analysis (Plaintext)}
\begin{quote}
Hereupon Legrand arose, with a grave and stately air, and brought me the beetle
from a glass case in which it was enclosed. It was a beautiful scarabaeus, and, at
that time, unknown to naturalists -- of course a great prize in a scientific point
of view. There were two round black spots near one extremity of the back, and a
long one near the other. The scales were exceedingly hard and glossy, with all the
appearance of burnished gold. The weight of the insect was very remarkable, and,
taking all things into consideration, I could hardly blame Jupiter for his opinion
respecting it.

\rightline{--Edgar Allan Poe's ``The Gold-Bug''}
\end{quote}
\end{frame}

\begin{frame}[fragile]\frametitle{Vigen\`{e}re (poly-alphabetic shift) Cipher}
\begin{itemize}
\item \textbf{Idea}: To ``smooth out'' the distribution in the ciphertext by mapping different instances of the same letter in the plaintext to different ones in the ciphertext
\item \textbf{Encryption}: $c_i=m_i+k_{[i\bmod t]}$, $t$ is the length (period) of $k$
\item \textbf{Cryptanalysis}: Need find $t$; if $t$ is known, need know whether the decryption ``makes sense'', but brute force ($26^t$) is infeasible for $t > 15$
\end{itemize}
\begin{exampleblock}{Example (Key is `cafe')}
\begin{description}[Ciphertext]
\item[Plaintext]  \verb|tellhimaboutme| \\
\item[Key]        \verb|cafecafecafeca| \\
\item[Ciphertext] \verb|??????????????| %\verb|WFRQKJSFEPAYPF|
\end{description}
\end{exampleblock}
\end{frame}
\begin{frame}[fragile]\frametitle{Kasiski's Method (to find $t$)}
\begin{itemize}
\item To identify repeated patterns of length 2 or 3
\item The distance between such appearances is a multiple of $t$
\item $t$ is the greatest common divisor of all the distances
\end{itemize}
\begin{exampleblock}{Example (Key is `beads')}
\begin{semiverbatim}
themanandthewomanretrievedtheletterfromthepostoffice
beadsbeadsbeadsbeadsbeadsbeansdeadsbeadsbeadsbeadbea
VMFQTPFOH\alert{MJJ}XSFCSSIMTNFZXFYISEIYUIKHWPQ\alert{MJJ}QSLVTGJKGF
\end{semiverbatim}
\end{exampleblock}
\end{frame}
\begin{frame}\frametitle{Index of Coincidence (IC) Method (to find $t$)}
For $\tau = 1, 2, \dotsc$, $q_i$ is the probability of $i$th letter in $c_1, c_{1+\tau}, c_{1+2\tau}, \dotsc$, IC is
\[I_\tau \overset{\text{def}}{=}\sum_{i=0}^{25} q_i^2\]
\alert{If $\tau = t$, then $I_\tau \approx ?$} ; otherwise $q_i \approx \frac{1}{26}$ and
\[I_\tau \approx \sum_{i=0}^{25} \left(\frac{1}{26}\right)^2 \approx 0.038\]
Then reuse IC method to find $k_i$.
\begin{alertblock}{Arbitrary Adversary Principle}
Security must be guaranteed for any adversary within the class of adversaries having the specified power
\end{alertblock}
\end{frame}
\begin{frame}\frametitle{Cryptanalytic Attacks (homework assignment)}
\begin{itemize}
\item Under COA, the requirement for ciphertext related to the size of the key space.  Vig\`{e}nere > mono-alphabetic sub. > shift
\item Under KPA, trivially broken.
\end{itemize}
\begin{alertblock}{Lessons learned}
\begin{itemize}
\item Sufficient key space principle
\item Designing secure cipher is a hard task
\item Complexity does not imply security (then what does?)
\item Arbitrary adversary principle
\end{itemize}
\end{alertblock}
\end{frame}
\section{The Basic Principles of Modern Cryptography}
\begin{frame}\frametitle{Three Main Principles of Modern Cryptography}
\begin{enumerate}
\item The formulation of a rigorous \textbf{definition} of security / threat model
\item When the security of a cipher relies on an unproven \textbf{assumption}, this assumption must be precisely stated and be as minimal as possible
\item Cipher should be accompanied by a rigorous \textbf{proof} of security with the above definition and the above assumption
\end{enumerate}
\end{frame}
\begin{frame}\frametitle{Why Principle 1 -- Formulation of Exact Definitions}
\begin{exampleblock}{Q: how would you formalize the security for private-key encryption?}
\begin{enumerate}
\item \emph{No adversary can find the secret key when given a ciphertext.}\\
$\mathsf{Enc}_k(m)=m$
\item \emph{No adversary can find the plaintext that corresponds to the ciphertext.}\\
$\mathsf{Enc}_k(m)=m_{0}\| \mathsf{AES}_k(m)$
\item \emph{No adversary can determine any character of the plaintext that corresponds to the ciphertext.}\\
$m=1000$, someone can learn $ 800 < m < 1200$
\item \emph{No adversary can derive any meaningful information about the plaintext from the ciphertext.}\\
Could you define so-called `meaningful'?
\end{enumerate}
\emph{\alert{Definitions of security should suffice for all potential applications.}}
\end{exampleblock}
\end{frame}
\begin{frame}\frametitle{Why Principle 1 -- How to define}
%\begin{exampleblock}{General Form}
%A cryptographic scheme for a given \textbf{task} is secure if no adversary of a specified \textbf{power} can achieve a specified \textbf{break}
%\end{exampleblock}

How To Define Security -- Lesson From Alan Turing
\begin{itemize}
\item What's computation?\footnote{Q: Any ``mathematical proof that there exist well-defined problems that computers cannot solve''? A: Halting Problem in computability theory}
\begin{enumerate}
\item A direct appeal to \textbf{intuition}
\item A \textbf{proof of the equivalence} of two definitions\\ (The new one has a greater intuitive appeal)
\item Giving \textbf{examples} solved using a definition
\end{enumerate}
\item Additional method for security: \textbf{Test of time}
\end{itemize}
\end{frame}	
\begin{frame}\frametitle{Principle 2 -- Reliance on Precise Assumptions}
Most cryptographic constructions \textbf{cannot be proven secure unconditionally}
\begin{itemize}
	\item \textbf{Why?} 
	\begin{enumerate}
		\item Validation of the assumption
		\item Comparison of schemes
		\item Facilitation of proofs of security
	\end{enumerate}
	\textbf{The construction is secure if the assumption is true.}
	\item \textbf{How?} 
	\begin{enumerate}
		\item old, so well tested
		\item simple and lower-level, so easy to study, refute \& correct
	\end{enumerate}
\end{itemize}
\end{frame}
\begin{frame}\frametitle{Principle 3 -- Rigorous Proofs of Security}
\begin{itemize}
\item \textbf{Why?} Proofs are more desirable in computer security than in other fields.
\item \textbf{The reductionist approach}: 
\begin{theorem}	Given that Assumption X is true, Construction Y is secure according to the given definition.
\end{theorem}
\begin{proof} Reduce the problem given by X to the problem of breaking Y.
\end{proof}
\item \textbf{Ad-hoc approaches}: for those who need a ``quick and dirty'' solution, or who are just simply unaware.
\end{itemize}
\end{frame}
\begin{frame}\frametitle{Summary}
\begin{itemize}
\item Cryptography secures information, transactions and computations
\item Kerckhoffs's principle \& Open cryptographic design
\item Caesar's, shift, Mono-Alphabetic sub., Vigen\`{e}re
\item Brute force, letter frequency, Kasiski's, IC
\item Sufficient key space principle
\item Arbitrary adversary principle
\item Rigorously proven security
\end{itemize}
\end{frame}
\begin{frame}\frametitle{What is cryptography? [xkcd:504]}
\begin{figure}
\begin{center}
\includegraphics[width=100mm]{pic/legal} 
\end{center}
\end{figure}
\end{frame}
\begin{frame}\frametitle{Alice, Bob  [xkcd:1323]}
Changing the names would be easier, but if you're not comfortable lying, try only making friends with people named Alice, Bob, Carol, etc.
\begin{figure}
\begin{center}
\includegraphics[width=45mm]{pic/alice-bob} 
\end{center}
\end{figure}
\end{frame}
\end{document}


%\input{2perfectlysecret.tex}
%\input{3privatekey.tex}


\title{Message Authentication Codes and Collision-Resistant Hash Functions}

\begin{document}
\maketitle
\begin{frame}
\frametitle{Outline}
\tableofcontents
\end{frame}
\section{Message Authentication Codes (MAC) -- Definitions}
\begin{frame}\frametitle{Integrity and Authentication}
\begin{figure}
\begin{center}
\input{tikz/integrity}
\input{tikz/authentication}
\end{center}
\end{figure}
\end{frame}
\begin{frame}\frametitle{The Syntax of MAC}
\begin{figure}
\begin{center}
\input{tikz/mac}
\end{center}
\end{figure}
\begin{itemize}
\item key $k$, tag $t$, a bit $b$ means $\mathsf{valid}$ if $b=1$; $\mathsf{invalid}$ if $b=0$.
\item \textbf{Key-generation} algorithm~$k \gets \mathsf{Gen}(1^n), \abs{k} \ge n$.
\item \textbf{Tag-generation} algorithm~$t \gets \mathsf{Mac}_k(m)$.
\item \textbf{Verification} algorithm~$b:= \mathsf{Vrfy}_k(m,t)$.
\item \textbf{Message authentication code}: $\Pi = (\mathsf{Gen}, \mathsf{Mac}, \mathsf{Vrfy})$.
\item \textbf{Basic correctness requirement}: $\mathsf{Vrfy}_k(m,\mathsf{Mac}_k(m)) = 1$.
\end{itemize}
\end{frame}
\begin{frame}\frametitle{Security of MAC}
\begin{itemize}
\item \textbf{Intuition}: No adversary should be able to generate a \textbf{valid} tag on any ``\textbf{new}'' message\footnote{A stronger requirement is concerning \emph{new message/tag pair}.} that was not previously sent.
\item \textbf{Replay attack}: Copy a message and tag previously sent. (\textbf{excluded by only considering ``new'' message})
\begin{itemize}
\item Sequence numbers: receiver must store the previous ones.
\item Time-Stamps: sender/receiver maintain synchronized clocks.
\end{itemize}
\item \textbf{Existential unforgeability}: \textbf{Not} be able to forge a valid tag on \textbf{any} message.
\begin{itemize}
\item \textbf{Existential forgery}: \emph{at least one} message.
\item \textbf{Selective forgery}: message chosen \emph{prior} to the attack.
\item \textbf{Universal forgery}: \emph{any} given message.
\end{itemize}
\item \textbf{Adaptive chosen-message attack (CMA)}: be able to obtain tags on \emph{any} message chosen adaptively \emph{during} its attack.
\end{itemize}
\end{frame}
\begin{frame}\frametitle{Definition of MAC Security}
The message authentication experiment $\mathsf{Macforge}_{\mathcal{A},\Pi }(n)$:
\begin{enumerate}
\item $k \gets \mathsf{Gen}(1^n)$.
\item $\mathcal{A}$ is given input $1^n$ and oracle access to $\mathsf{Mac}_k(\cdot)$, and outputs $(m,t)$. $\mathcal{Q}$ is the set of queries to its oracle.
\item $\mathsf{Macforge}_{\mathcal{A},\Pi }(n)=1 \iff$ $\mathsf{Vrfy}_k(m,t)=1$ $\land$ $m \notin \mathcal{Q}$. 
\end{enumerate}
\begin{figure}
\begin{center}
\input{tikz/macforge-exp.tex}
\end{center}
\end{figure}
\begin{definition}
A MAC $\Pi$ is \textbf{existentially unforgeable under an adaptive CMA} if $\forall$ \textsc{ppt} $\mathcal{A}$, $\exists$ $\mathsf{negl}$ such that:
\[ \Pr [\mathsf{Macforge}_{\mathcal{A},\Pi }(n)=1] \le \mathsf{negl}(n).
\]
\end{definition}
\end{frame}
\begin{frame}\frametitle{Questions}
\begin{exampleblock}{Suppose $\left<S, V\right>$ are CMA-secure,  are $\left<S', V'\right>$ secure?}
\begin{itemize}
%\item Suppose an attacker can find $m_{0} \neq m_{1}$ s.t. $t_{0} = t_{1}$ for $\frac{1}{8}$.
%\item Suppose tag is always $32$ bits long.
\item $S'_{k_{1},k_{2}}(m) = (S_{k1}(m),S_{k_{2}}(m))$\\
$V'_{k_{1},k_{2}}(m,(t_{1},t_{2})) = V_{k1}(m,t_{1}) \land V_{k_{2}}(m,t_{2})$
\item $S'_{k}(m) = (S_{k}(m),S_{k}(m))$\\
$ V'_{k}(m,(t_{1},t_{2})) = \left\{ 
  \begin{array}{l l}
    V_{k}(m,t_{1}) & \quad \text{if $t_{1}=t_{2}$}\\
    0 & \quad \text{otherwise}\\
  \end{array} \right. $
\item $S'_{k}(m) = (S_{k}(m),S_{k}(0^{n}))$\\
$ V'_{k}(m,(t_{1},t_{2})) = V_{k}(m,t_{1}) \land  V_{k}(0^{n},t_{2})$
\item $S'_{k}(m) = S_{k}(m)$, 
$ V'_{k}(m,t) = \left\{ 
  \begin{array}{l l}
    V_{k}(m,t) & \quad \text{if $m \neq 0^{n}$}\\
    1 & \quad \text{otherwise}\\
  \end{array} \right. $
\item $S'_{k}(m) = S_{k}(m)\ \text{without the LSB}$ \\
$V'_{k}(m,t) = V_{k}(m,t\| 0)\ \lor \ V_{k}(m,t\| 1)$
\item $S'_{k}(m) = (S_{k}(m),m)$, 
$V'_{k}(m,(t_{1},t_{2})) = V_{k}(m,t_{1}) \land t_{2} = m$
\end{itemize}
\end{exampleblock}
\end{frame}
\section{Constructing Secure Message Authentication Codes}
\begin{frame}\frametitle{Constructing Secure MACs}
\begin{columns}[c]
\column{.4\textwidth}
\begin{figure}
\begin{center}
\input{tikz/macwithprf}
\end{center}
\end{figure}
\column{.6\textwidth}
\begin{construction}
\begin{itemize}
\item $F$ is PRF. $\abs{m} = n$.
\item $\mathsf{Gen}(1^n)$: $k \gets \{0,1\}^n$ \emph{u.a.r}.
\item $\mathsf{Mac}_k(m)$: $t := F_k(m)$.
\item $\mathsf{Vrfy}_k(m,t)$: $1 \iff t \overset{?}{=} F_k(m)$.
\end{itemize}
\end{construction}
\begin{theorem}\label{thm:mac}
If $F$ is a PRF, Construction is a secure fixed-length MAC.
\end{theorem}
\end{columns}
\begin{lemma}
\textbf{Truncating MACs based on PRFs}:
If $F$ is a PRF, so is $F^t_k(m) = F_k(m)[1,\dots,t]$.
\end{lemma}
\end{frame}
\begin{frame}\frametitle{Proof of Secure MAC from PRF}
\textbf{Idea}: Show $\Pi$ is secure unless $F_k$ is not PRF by reduction.  
\begin{proof}
$D$ distinguishes $F_k$; $\mathcal{A}$ attacks $\Pi$. 
\begin{figure}
\begin{center}
\input{tikz/pgfMAC}
\end{center}
\end{figure}
\end{proof}
\end{frame}
\begin{frame}\frametitle{Proof of Secure MAC from PRF (Cont.)}
\begin{proof}
(1) If true random $f$ is used, $t=f(m)$ is uniformly distributed.
\[ \Pr[D^{f(\cdot)}(1^n)=1] = \Pr[\mathsf{Macforge}_{\mathcal{A},\tilde{\Pi}}(n) = 1] \le 2^{-n}.\]
(2) If $F_k$ is used, conduct the experiment $\mathsf{Macforge}_{\mathcal{A},\Pi}(n)$. 
\[ \Pr[D^{F_k(\cdot)}(1^n)=1] = \Pr[\mathsf{Macforge}_{\mathcal{A},\Pi}(n) = 1] = \varepsilon(n).\]
According to the definition of PRF,
\[ \left| \Pr[D^{F_k(\cdot)}(1^n)=1] - \Pr[D^{f(\cdot)}(1^n)=1] \right| \ge \varepsilon(n) - 2^{-n}. \]
\end{proof}
\end{frame}
\begin{frame}\frametitle{Extension to Variable-Length Messages}
\begin{exampleblock}{For variable-length messages, would the following suggestions be secure?}
\begin{itemize}
\item \textbf{Suggestion 1}: XOR all the blocks together and authenticate the result. $t := \mathsf{Mac}_k'(\oplus_i m_i)$.
\item \textbf{Suggestion 2}: Authenticate each block separately. $t_i := \mathsf{Mac}_k'(m_i)$.
\item \textbf{Suggestion 3}: Authenticate each block along with a sequence number. $t_i := \mathsf{Mac}_k'(i\| m_i)$.
%\item \textbf{Weakness}: forgeable, changing the order, dropping blocks.
\end{itemize}
\end{exampleblock}
%\item \textbf{Countermeasure}: add information. 
%\begin{itemize}
%\item random ``\textbf{message identifier}'' provides randomness; prevents combination.
%\item \textbf{sequence number} prevents reordering.
%\item the \textbf{length} of message prevents dropping/appending.
%\end{itemize}
%\end{itemize}
\end{frame}
\begin{comment}
\begin{frame}\frametitle{Constructing Secure Variable-Length MACs}
\begin{construction}
\begin{itemize}
\item $\Pi' = (\mathsf{Gen}', \mathsf{Mac}', \mathsf{Vrfy}')$ be a fixed-length MAC.
\item $\mathsf{Gen}$: is identical to $\mathsf{Gen}'$.
\item $\mathsf{Mac}$: $m$ of length $\ell < 2^{n/4}$ and of $d$ blocks $m_1,\dotsc,m_d $ of length $n/4$ (padded with 0s); $r \gets \{0,1\}^{n/4}$.\\
For $i=1,\dotsc,d$, $t_i \gets \mathsf{Mac}_k'(r\| \ell\| i\| m_i)$, $i$ and $\ell$ are uniquely encoded as strings of length $n/4$.\\
Output $t:=\left<r,t_1,\dotsc,t_d\right>$.
\item $\mathsf{Vrfy}$: Input $m$ of $d'$ blocks and check $d'=d$.\\
Output $1 \iff \mathsf{Vrfy}_k'(r\| \ell\| i\| m_i, t_i)=1$ for $1\le i \le d$.
\end{itemize}
\end{construction}
\begin{theorem}
If $\Pi'$ is a secure fixed-length MAC, Construction is a secure MAC.
\end{theorem}
\end{frame}
\begin{frame}\frametitle{Proof of Secure Variable-Length MACs}
\textbf{Intuition}: The extra information prevents all possible attacks.
\begin{proof}
\begin{description}
\item[$\mathsf{Repeat}$]: the same identifier $r$ is used twice by oracle $\mathcal{O}$. 
\item[$\mathsf{Forge}$]: at least one new block $r\| \ell\| i\| m_i$ is forged. 
\item[$\mathsf{Break}$]: $\mathsf{Macforge}_{\mathcal{A},\Pi }(n)=1, \Pr[\mathsf{Break}]=\varepsilon(n)$. 
\end{description}
\[
\begin{split}
	\Pr[\mathsf{Break}] =& \Pr[\mathsf{Break} \land \mathsf{Repeat}] + \Pr[\mathsf{Break} \land \overline{\mathsf{Repeat}} \land \overline{\mathsf{Forge}}] \\
	&+ \Pr[\mathsf{Break} \land \overline{\mathsf{Repeat}} \land \mathsf{Forge}].
\end{split}
\]
To prove the below statements:
\begin{enumerate}
\item $\Pr[\mathsf{Break} \land \mathsf{Repeat}] \le \Pr[\mathsf{Repeat}] \le \mathsf{negl}(n)$.
\item $\Pr[\mathsf{Break} \land \overline{\mathsf{Repeat}} \land \overline{\mathsf{Forge}}] = 0$.
\item For $\Pi'$, $\Pr[\mathsf{Break}'] = \Pr[\mathsf{Break} \land \mathsf{Forge}] \ge \Pr[\mathsf{Break} \land \overline{\mathsf{Repeat}} \land \mathsf{Forge}] \ge \varepsilon(n) - \mathsf{negl}(n)$.
\end{enumerate}
\end{proof}
\end{frame}
\begin{frame}\frametitle{Proof of Secure Variable-Length MACs (Cont.)}
\begin{proof}
\begin{enumerate}
\item $r \gets \{0,1\}^{\frac{n}{4}}$. By ``brithday bound'', $\Pr[\mathsf{Repeat}] \le q(n)^2/2^{\frac{n}{4}}$.
\item If $\mathsf{Repeat}$ does not occur, $\mathsf{Break}$ implies $\mathsf{Forge}$. \\
$\mathcal{A}$ finally outputs $(m,t), t:=\left<r,t_1,\dotsc,t_d\right>$.
\begin{itemize}
\item $r$ is new, then $r\| \ell\| i\| m_i$ is new.
\item $r$ is used exactly once, then the queried message $m' \neq m$. 
\begin{itemize}
\item $\ell' \neq \ell$, then $r\| \ell\| i\| m_i$ is new.
\item $\ell' = \ell$, then $\exists\; m_i' \neq m_i$, so $r\| \ell\| i\| m_i'$ is new.
\end{itemize}
\end{itemize}
So the block is new, $\mathsf{Forge}$ occurs.
\item Reduce $\mathcal{A}'$ to $\mathcal{A}$: $\mathcal{A}'$ attacks $\Pi'$ with $\mathcal{A}$ as a sub-routine and answer the queries of $\mathcal{A}$ with $\mathcal{A}'$'s own oracle. $\mathcal{A}$ output $(m,t)$; $\mathcal{A}'$ parses it and output a new block $(r\| \ell\| i\| m_i, t_i)$ if possible.
\end{enumerate}
\end{proof}
\end{frame}
\end{comment}
\section{CBC-MAC}
\begin{frame}\frametitle{Constructing Fixed-Length CBC-MAC}
\begin{columns}[c]
\column{.5\textwidth}
\begin{figure}
\begin{center}
\input{tikz/CBC-small}
\end{center}
\end{figure}
\column{.5\textwidth}
\begin{figure}
\begin{center}
\input{tikz/CBC-MAC}
\end{center}
\end{figure}
\end{columns}
Modify CBC encryption into CBC-MAC:
\begin{itemize}
\item Change random $IV$ to encrypted fixed $0^{n}$,\emph{otherwise}:\\
\alert{Q: query $m_1$ and get $(IV, t_1)$; output $m_1' = IV' \oplus  IV \oplus m_{1}$ and $t' =$ \underline{$\qquad $}.} %(IV',t_1)$.
\item Tag only includes the output of the final block,\emph{otherwise}:\\
\alert{Q: query $m_i$ and get $t_i$; output $m_i' = t_{i-1}' \oplus t_{i-1} \oplus m_{i}$ and $t_{i}' = $ \underline{$\qquad$}.}%$t_i$.
\end{itemize}
\end{frame}
\begin{frame}\frametitle{Constructing Fixed-Length CBC-MAC (Cont.)}
\begin{construction}
\begin{itemize}
\item a PRF $F$ and a length function $\ell$. $\abs{m} = \ell(n)\cdot n$.
$\ell=\ell(n)$. $m = m_1,\dotsc,m_{\ell}$.
\item $\mathsf{Gen}(1^n)$: $k \gets \{0,1\}^n$ \emph{u.a.r}.
\item $\mathsf{Mac}_k(m)$: $t_i := F_k(t_{i-1}\oplus m_i), t_0=0^n$. Output $t = t_\ell$.
\item $\mathsf{Vrfy}_k(m,t)$: $1 \iff t \overset{?}{=} \mathsf{Mac}_k(m)$.
\end{itemize}
\end{construction}
\begin{theorem}
If $F$ is a PRF, Construction is a secure \textbf{fixed-length} MAC.
\end{theorem}
\textbf{Not} for \textbf{variable-length} message:\\
\alert{Q: For one-block message $m$ with tag $t$, adversary can append a block \underline{$\qquad$} and output tag $t$.} %$t\oplus m$
\end{frame}
\begin{frame}\frametitle{Secure Variable-Length MAC}
\begin{itemize}
\item \textbf{Input-length key separation}: $k_{\ell} := F_k(\ell)$, use $k_{\ell}$ for CBC-MAC.
\item \textbf{Length-prepending}: Prepend $m$ with $|m|$, then use CBC-MAC.
\begin{figure}
\begin{center}
\input{tikz/VCBC-MAC}
\end{center}
\end{figure}
\item \textbf{Encrypt last block (ECBC-MAC)}: Use two keys $k_1, k_2$. Get $t$ with $k_1$ by CBC-MAC, then output $\hat{t} := F_{k_2}(t)$.
\end{itemize}
\alert{Q: To authenticate a voice stream, which approach do you prefer?}
\end{frame}
\begin{comment}
\begin{frame}{Brute-force Attack against CBC-MAC}
Query $2^{\abs{t}/2}$ message to find $m \neq m'$ and $t = t'$.
\newline

\textbf{Extension property} of ECBC-MAC:
\[ \forall x,y,z: F_k(x)=F_k(y) \Rightarrow F_k(x\|z)=F_k(y\|z).   \]

So the tag of $m\|w$ is the same with that of $m'\|w$.
\newline

Lesson: the tag space should be enough large.\\
Improvement: Add a random string $r$, and output $(r, \mathsf{Mac}_{k'}(t\|r))$ instead of $t$.
\end{frame}
\end{comment}
\begin{frame}\frametitle{MAC Padding}
Padding must be invertible!\[ m_0\neq m_1 \Rightarrow \mathsf{pad}(m_0) \neq \mathsf{pad}(m_1). \]
\textbf{ISO}: pad with ``100\dots00''. Add dummy block if needed.\\
\alert{Q: What if no dummy block?} \\
\textbf{CMAC (Cipher-based MAC from NIST)}: key$=(k,k_1,k_2)$.
\begin{figure}
\begin{center}
\input{tikz/CMAC}
\end{center}
\end{figure}
\begin{itemize}
\item No final encryption: extension attack thwarted by keyed XOR.
\item No dummy block: ambiguity resolved by use of $k_1$ or $k_2$.
\end{itemize}
\end{frame}
\section{Collision-Resistant Hash Functions}
\begin{frame}\frametitle{Defining Hash Function}
\begin{figure}
\begin{center}
\input{tikz/hash}
\end{center}
\end{figure}
\begin{definition}
A \textbf{hash function (compression function)} is a pair of \textsc{ppt} algorithms $(\mathsf{Gen}, H)$ satisfying:
\begin{itemize}
\item a key $s \gets \mathsf{Gen}(1^n)$, $s$ is \textbf{not kept secret}.
\item $H^s(x) \in \{0,1\}^{\ell(n)}$, where $x \in \{0,1\}^*$ and $\ell$ is polynomial.
\end{itemize}
If $H^s$ is defined only for $x \in \{0,1\}^{\ell'(n)}$ and $\ell'(n) > \ell(n)$, then $(\mathsf{Gen}, H)$ is a \textbf{fixed-length} hash function.
\end{definition}
\end{frame}
\begin{frame}\frametitle{Defining Collision Resistance}
\begin{itemize}
\item \textbf{Collision} in $H$: $x \neq x'$ and $H(x) = H(x')$.
\item \textbf{Collision Resistance}: infeasible for any \textsc{ppt} alg. to find.
\end{itemize}
The collision-finding experiment $\mathsf{Hashcoll}_{\mathcal{A},\Pi}(n)$:
\begin{enumerate}
\item $s \gets \mathsf{Gen}(1^n)$.
\item $\mathcal{A}$ is given $s$ and outputs $x, x'$.
\item $\mathsf{Hashcoll}_{\mathcal{A},\Pi}(n) =1 \iff x\ne x' \land H^s(x) = H^s(x')$.
\end{enumerate}
\begin{definition}
$\Pi$ ($\mathsf{Gen}$, $H^s$) is \textbf{collision resistant} if $\forall$ \textsc{ppt} $\mathcal{A}$, $\exists\;\mathsf{negl}$ such that
\[ \Pr[\mathsf{Hashcoll}_{\mathcal{A},\Pi}(n)=1] \le \mathsf{negl}(n).
\]
\end{definition}
\end{frame}
%\begin{comment}
\begin{frame}\frametitle{Weaker Notions of Security for Hash Functions}
\begin{figure}
\begin{center}
\input{tikz/collision}
\end{center}
\end{figure}
\begin{itemize}
\item \textbf{Collision resistance}: It is hard to find $(x, x'), x' \ne x$ such that $H(x) = H(x')$.
\item \textbf{Second pre-image resistance}: Given $s$ and $x$, it is hard to find $x' \ne x$ such that $H^s(x') = H^s(x)$.
\item \textbf{Pre-image resistance}: Given $s$ and $y = H^s(x)$, it is hard to find $x'$ such that $H^s(x')=y$.
\end{itemize}
\end{frame}
\begin{frame}{Questions}
\begin{exampleblock}{$H$ is CRHF. Is $H'$ CRHF?}
\begin{itemize}
\item $H'(m) = H(m)\oplus H(m\oplus 1^{\abs{m}})$ % H(000) = H(111)
\item $H'(m) = H(m)\| H(0)$
\item $H'(m) = H(m)\| H(m)$
\item $H'(m) = H(m) \oplus H(m)$
\item $H'(m) = H(m[0,\dots,\abs{m}-2])$
\item $H'(m) = H(m\| 0)$
\end{itemize}
\end{exampleblock}
\end{frame}
\begin{frame}{Applications of Hash Functions}
\begin{itemize}
\item \textbf{Fingerprinting and Deduplication}: $H(alargefile)$ for virus fingerprinting, deduplication, P2P file sharing
\item \textbf{Merkle Trees}: $H(H(H(file1), H(file2)), H(H(file3), H(file4)))$ fingerprinting multiple files / parts of a file
\item \textbf{Password Hashing}: $(salt, H(salt, pw))$ mitigating the risk of leaking password stored in the clear 
\item \textbf{Key Derivation}: $H(secret)$ deriving a key from a high-entropy (but not necessarily uniform) shared secret
\item \textbf{Commitment Schemes}: $H(info)$ hiding the commited info; binding the commitment to a info
\end{itemize}
\end{frame}
%\end{comment}
%\begin{frame}\frametitle{Applications of Hash Functions}
%\begin{itemize}
%\item \textbf{digital signatures}:CRHF
%\item \textbf{information authentication/integrity check}
%\item \textbf{protection of passwords}: pre-image resistant.
%\item \textbf{confirmation of knowledge/commitment}: CRHF
%\item \textbf{pseudo-random string generation/key derivation}
%\item \textbf{micropayments (e.g. micromint)}
%\item \textbf{construction of MACs, stream/block ciphers}
%\end{itemize}
%\end{frame}
\begin{frame}\frametitle{The ``Birthday'' Problem}
\begin{exampleblock}{The ``Birthday'' Problem}
\textbf{Q}: ``\emph{What size group of people do we need to take such that with probability $1/2$ some pair of people in the group share a birthday?}''

\textbf{A}: 23.
\end{exampleblock}
\begin{lemma}
Choose $q$ elements $y_1,\dotsc , y_q$ \emph{u.a.r} from a set of size $N$, the probability that $\exists \; i \ne j$ with $y_i = y_j$ is $\mathsf{coll}(q,N)$, then 
\[ \mathsf{coll}(q,N) \le \frac{q^2}{2N}.
\]
\[ \mathsf{coll}(q,N) \ge  \frac{q(q-1)}{4N}\quad \text{if}\; q \le \sqrt{2N}.
\]
\[ \mathsf{coll}(q,N) = \Theta(q^2/N)\quad \text{if}\; q < \sqrt{N}.
\]
\end{lemma}
The length of hash value should be long enough.
\end{frame}
%\begin{frame}\frametitle{A Generic ``Birthday'' Attack}
%\begin{itemize}
%\item \textbf{Birthday Attack}: $H : \{0,1\}^* \to \{0,1\}^\ell$. Choose $q$ distinct inputs $x_1,\dotsc,x_q \in \{0,1\}^{2\ell}$, check whether any of two $y_i := H(x_i)$ are equal.
%\item \textbf{Birthday problem}: Choose $y_1,\dotsc,y_q \gets \{0,1\}^{\ell}$ \emph{u.a.r}, $\mathsf{coll}(q,2^{\ell}) = ?$
%\item Collision occurs with a high probability when $\mathcal{O}(q) = \mathcal{O}(2^{\ell/2})$.
%\item To let time $T > 2^{\ell/2}$, then $\ell = 2\log T$ at least.
%\item Work only for collision resistance, no generic attacks for 2nd pre-image or pre-image resistance better than $2^\ell$.
%\item Require too much space $\mathcal{O}(2^{\ell/2})$.
%\end{itemize}
%\end{frame}
\begin{comment}
\begin{frame}\frametitle{Improved Birthday Attack}
\begin{algorithm}[H]
\SetKwInOut{Input}{input}
\SetKwInOut{Output}{output}
\SetKw{KwB}{break}
\SetKw{KwH}{halt}
\DontPrintSemicolon
\caption{Improved birthday attack}
\Input{A hash function $H : \{0, 1\}^*\to \{0, 1\}^\ell$}
\Output{Distinct $x, x'$ with $H(x) = H(x')$}
\BlankLine
$x_0 \gets \{0,1\}^{\ell+1}$, $x' := x := x_0$\;
\For{$i = 1$ \KwTo $2^{\ell/2} +1$}{
  $x := H(x)$, $x' := H(H(x'))$
  \tcp{$x = H^i(x_0)$, $x' = H^{2i}(x_0)$}
  \lIf{$x=x'$}{\KwB}\;
}
\lIf{$x\ne x'$}{\Return fail}\;
$x' := x$, $x := x_0$\;
\For{$j=1$ \KwTo $i$}{
  \lIf{$H(x)=H(x')$}{\Return $x, x'$ and \KwH}\;
  \lElse{$x := H(x), x' := H(x')$}
  \tcp{$x = H^j(x_0)$, $x' = H^{j+i}(x_0)$}
}
\end{algorithm}
\end{frame}
\begin{frame}\frametitle{Proof of Improved Birthday Attack}
\begin{lemma}
Let $x_1,\dotsc,x_q$ be a sequence of values with $x_m = H(x_{m-1})$. If $x_I =x_J$ with $I < J$, then $\exists\; i<J$ such that $x_i =x_{2i}$.
\end{lemma}
\begin{figure}
\begin{center}
\input{tikz/birthdayattack}
\end{center}
\end{figure}
\begin{proof}
If $x_I =x_J$, then $x_I, x_{I+1}, \dotsc$ repeats with period $J-I$.\\
Let $i$ to be the smallest multiple of $J-I$ with $i \ge I$, \[i \overset{\text{def}}{=} (J-I)\cdot \lceil I/(J-I)\rceil.\] \\
$i < J$ since $I,\dotsc,J-1$ contains a multiple of $J-I$.\\
Since $2i-i=i$ is a multiple of the period and $i \ge I$, $x_i = x_{2i}$. 
\end{proof}
\end{frame}
\end{comment}
\begin{frame}\frametitle{Constructing ``Meaningful'' Collisions}
\begin{exampleblock}{How many different meaningful sentences are in the below paragraph?}
It is \textbf{hard/difficult/challenging/impossible} to \textbf{imagine/believe} that we will \textbf{find/locate/hire} another \textbf{employee/person} having similar \textbf{abilities/skills/character} as Alice. She has done a \textbf{great/super} job.
\end{exampleblock}
\end{frame}
\begin{frame}\frametitle{The Merkle-Damg\r{a}rd Transform}
\begin{figure}
\begin{center}
\input{tikz/MDtransform}
\end{center}
\end{figure}
\begin{construction}
Construct \textbf{variable-length} CRHF $(\mathsf{Gen}, H)$ from fixed-length $(\mathsf{Gen}, h)$ ($2\ell$ bits $\to \ell$ bits, $\ell = \ell(n)$):
\begin{itemize}
\item $\mathsf{Gen}$: remains unchanged
\item $H$: key $s$ and string $x \in \{0,1\}^*$, $L=|x|< 2^{\ell}$:
\begin{itemize}
\item $B := \lceil \frac{L}{\ell} \rceil$ (\# blocks). \textbf{Pad $x$ with 0s}.  $\ell$-bit blocks $x_1,\dotsc,x_B$. $x_{B+1} := L$, $L$ is encoded using $\ell$ bits
\item $z_0 := IV = 0^\ell$. For $i=1,\dotsc,B+1$, compute $z_i := h^s(z_{i-1}\| x_i)$
\end{itemize}
\end{itemize}
\end{construction}
\end{frame}
\begin{frame}\frametitle{Security of the Merkle-Damg\r{a}rd Transform}
\begin{theorem}
If $(\mathsf{Gen},h)$ is a fixed-length CRHF, then $(\mathsf{Gen},H)$ is a CRHF.
\end{theorem}
\begin{proof}
\textbf{Idea}: a collision in $H^s$ yields a collision in $h^s$. \\
Two messages $x \ne x'$ of respective lengths $L$ and $L'$ such that $H^s(x) = H^s(x')$. \# blocks are $B$ and $B'$. \\
$x_{B+1} := L$ is necessary since \textbf{Padding with 0s} will lead to the same input with different messages.
\begin{enumerate}
\item $L \ne L'$: $z_B\| L \ne z_{B'}\| L'$
\item $L = L'$: $z_{i^*-1}\| x_{i^*} \ne z_{i^*-1}'\| x_{i^*}'$
\end{enumerate}
So there must be $x \neq x'$ such that $h^s(x) = h^s(x')$.
\end{proof}
Security on MD transform variations in Homework.
\end{frame}
\begin{frame}\frametitle{CRHF from Block Cipher}
\begin{columns}
\column{.5\textwidth}
Davies-Meyer (SHA-1/2, MD5)
\begin{figure}
\begin{center}
\input{tikz/Davies-Meyer.tex}
\end{center}
\end{figure}
$h_{i} = F_{m_{i}}(h_{i-1}) \oplus h_{i-1}$
\column{.5\textwidth}
Miyaguchi-Preneel (Whirlpool)
\begin{figure}
\begin{center}
\input{tikz/Miyaguchi-Preneel.tex}
\end{center}
\end{figure}
$h_{i} = F_{h_{i-1}}(m_{i}) \oplus h_{i-1} \oplus m$
\end{columns}
$\quad$\\
\begin{theorem}
If $F$ is modeled as an ideal cipher, then Davies-Meyer construction yields a CRHF.
\end{theorem}
\alert{Q: what if $h_{i} = F_{m_{i}}(h_{i-1})$ without XOR with $h_{i-1}$? }\\
\alert{Q: what if $F$ is not ideal such that $\exists x, F_k(x)=x$?}
\end{frame}
\begin{frame}\frametitle{Cryptographic Hash Functions: SHA-1 and MD5}
\begin{columns}[c]
\column{.5\textwidth}
SHA-1:
\begin{figure}
\begin{center}
\includegraphics[width=40mm]{pic/SHA1}
\end{center}
\end{figure}
\column{.5\textwidth}
MD5:
\begin{figure}
\begin{center}
\includegraphics[width=40mm]{pic/MD5}
\end{center}
\end{figure}
\end{columns}
$A, B, C, D$ and $E$ are 32-bit words of the state;
$F$ is a nonlinear function that varies;
$\lll n$ denotes a left bit rotation by $n$ places;
$W_t$/$M_t$ is the expanded message word of round $t$;
$K_t$ is the round constant of round $t$;
$\boxplus$ denotes addition modulo $2^{32}$.
\begin{itemize}
\item Finding a collision in 128-bit MD5 requires time $2^{20.96}$
\item Finding a collision in 160-bit SHA-1 requires time $2^{51}$
\end{itemize}
\end{frame}
\section{HMAC}
%\begin{frame}\frametitle{Nested MAC (NMAC)}
%\begin{figure}
%\begin{center}
%\input{tikz/NMAC}
%\end{center}
%\end{figure}
%\begin{construction}
%$(\widetilde{\mathsf{Gen}}, h)$ is a fixed-length CRHF. $(\widetilde{\mathsf{Gen}}, H)$ is Merkle-Damg\r{a}rd transform. NMAC:
%\begin{itemize}
%\item $\mathsf{Gen}(1^n)$: Output $(s, k_1, k_2)$. $s \gets \widetilde{\mathsf{Gen}}, k_1,k_2 \gets \{0,1\}^n$ \emph{u.a.r}.
%\item $\mathsf{Mac}_{s,k_1,k_2}(m)$: $t_i := h_{k_1}^s(H_{k_2}^s(m))$. $h_{k}^s \overset{\mathsf{def}}{=} h^s(k\|x)$.\\
%$H^s_{k_2}$ is \emph{inner} function; $h^s_{k_1}$ is \emph{outer} function.
%\item $\mathsf{Vrfy}_{s,k_1,k_2}(m,t)$: $1 \iff t \overset{?}{=} \mathsf{Mac}_{s,k_1,k_2}(m)$.
%\end{itemize}
%\end{construction}
%\end{frame}
%\begin{frame}\frametitle{Security of NMAC}
%\begin{theorem}
%If $(\widetilde{\mathsf{Gen}}, h)$ is CRHF and yields a secure MAC, then NMAC is secure. (existentially unforgeable under an adaptive CMA for arbitrary-length messages)
%\end{theorem}
%\begin{itemize}
%\item $k_2$ is not needed once $(\widetilde{\mathsf{Gen}}, h)$ is CRHF.
%\begin{itemize}
%\item \textbf{Weak collision resistance}: It is hard to find $(x, x'), x' \ne x$ such that $H^s_{k_2}(x) = H^s_{k_2}(x')$.
%\item $H_s^{k_2}(x)$ is hidden by $h_s^{k_1}(H_s^{k_2}(x))$.
%\item \textbf{Disadvantage}: $IV$ of $H$ must be modified.
%\end{itemize}
%\end{itemize}
%\end{frame}
\begin{frame}\frametitle{Hash-based MAC (HMAC)}
\begin{figure}
\begin{center}
\input{tikz/HMAC}
\end{center}
\end{figure}
\begin{construction}
$(\widetilde{\mathsf{Gen}}, h)$ is a fixed-length CRHF. $(\widetilde{\mathsf{Gen}}, H)$ is the Merkle-Damg\r{a}rd transform.
$IV$, $\mathsf{opad}$ (0x36), $\mathsf{ipad}$ (0x5C) are fixed constants of length $n$.
HMAC:
\begin{itemize}
\item $\mathsf{Gen}(1^n)$: Output $(s, k)$. $s \gets \widetilde{\mathsf{Gen}}, k \gets \{0,1\}^n$ \emph{u.a.r}
\item $\mathsf{Mac}_{s,k}(m)$: $t := H_{IV}^s\Big((k \oplus \mathsf{opad}) \| H_{IV}^s\big((k \oplus \mathsf{ipad}) \| m\big)\Big)$
\item $\mathsf{Vrfy}_{s,k}(m,t)$: $1 \iff t \overset{?}{=} \mathsf{Mac}_{s,k}(m)$
\end{itemize}
\end{construction}
\end{frame}
\begin{frame}\frametitle{Security of HMAC}
\begin{theorem} \[ G(k) \overset{\text{def}}{=} h^s(IV\| (k\oplus \mathsf{opad})) \| 
h^s(IV\| (k\oplus \mathsf{ipad})) = k_1\| k_2
\]
$(\widetilde{\mathsf{Gen}}, h)$ is CRHF. If $G$ is a PRG, then HMAC is secure.
\end{theorem}
Before HMAC's $H^s(k \| H^s(k \| x))$, a common mistake was to use:
\begin{itemize}
\item $H^s(k\| x)$: Vulnerable to length extension attack. Given $H^s(k\| x)$ and the length of $x$, get the valid tag $H^s(k\| x \| x')$ for a new message $x \| x'$.
\item $H^s(x\| k)$: A collision in the weak hash function has a collision in the MAC. 
\item $H^s(k\| x\| k)$: Some known vulnerabilities with this approach, even when two different keys are used.
\end{itemize}
\end{frame}
\begin{frame}\frametitle{Security of HMAC (Cont.)}
\begin{itemize}
\item HMAC is an industry standard (RFC2104)
\item HMAC is faster than CBC-MAC
\item \alert{Verification timing attacks: (Keyczar crypto library (Python))} \\
def Verify(key, msg, sig\underline{\ }bytes): \\
$\qquad$ return HMAC(key, msg) == sig\underline{\ }bytes \\
The problem:  implemented as a byte-by-byte comparison
\item \alert{\emph{Don't implement it yourself}}
\end{itemize}
\end{frame}
\section{Information-Theoretic MACs}

\begin{frame}\frametitle{Definition of Information-Theoretic MAC Security}
It is impossible to achieve "perfect" MAC, as the adversary can output a valid tag with probability $1/2^{\abs{t}}$ at least.
\newline

The one-time MAC experiment $\mathsf{Macforge}^{\mathsf{1-time}}_{\mathcal{A},\Pi }$:
\begin{enumerate}
\item $k \gets \mathsf{Gen}$.
\item $\mathcal{A}$ outputs a message $m'$, and is given a tag $t' \gets \mathsf{Mac}_k(m')$, and outputs $(m,t)$.
\item $\mathsf{Macforge}^{\mathsf{1-time}}_{\mathcal{A},\Pi }=1 \iff$ $\mathsf{Vrfy}_k(m,t)=1$ $\land$ $m \neq m'$. 
\end{enumerate}
\begin{definition}
A MAC $\Pi$ is \textbf{one-time $\varepsilon$-secure} if $\forall$ \textsc{ppt} $\mathcal{A}$:
\[ \Pr [\mathsf{Macforge}^{\mathsf{1-time}}_{\mathcal{A},\Pi}=1] \le \varepsilon.
\]
\end{definition}
\end{frame}

\begin{frame}\frametitle{Construction of Information-Theoretic MACs}
\begin{definition}
A function $h$: $\mathcal{K} \times \mathcal{M} \to \mathcal{T}$ is a \textbf{Strongly Universal Function (SUF)} if for all distinct $m, m' \in \mathcal{M}$ and all $t, t' \in \mathcal{T}$, it holds that:
\[ \Pr [h_k(m) = t  \land h_k(m') = t'] = 1 / \abs{\mathcal{T}}^2.
\]
where the probability is taken over uniform choice of $k \in \mathcal{K}$.
\end{definition}
\begin{construction}
\begin{itemize}
\item Let $h$: $\mathcal{K} \times \mathcal{M} \to \mathcal{T}$ be an SUF.
\item $\mathsf{Gen}$: $k \gets \{0,1\}^n$ \emph{u.a.r}.
\item $\mathsf{Mac}_k(m)$: $t := h_k(m)$.
\item $\mathsf{Vrfy}_k(m,t)$: $1 \iff t \overset{?}{=} h_k(m)$. (If $m \notin \mathcal{M}$, then output 0.)
\end{itemize}
\end{construction}
\end{frame}
\begin{frame}\frametitle{Construction of An SUF}
\begin{theorem}
For any prime $P$, the function $h$ is an SUF:
\[ h_{a,b}(m) \overset{\mathsf{def}}{=} [ a \cdot m + b \mod p]
\]
\end{theorem}
\begin{proof}
$h_{a,b}(m) = t$ and $h_{a,b}(m') = t'$, only if 
$a \cdot m + b  = t \mod p$  and  $a \cdot m' + b = t' \mod p$. We have $a = [(t-t') \cdot (m - m')^{-1} \mod p]$ and $b = [t - a \cdot m \mod p]$, which means there is a unique key $(a, b)$. Since there are $\abs{\mathcal{T}}^2$ keys, 
\[ \Pr [h_k(m) = t  \land h_k(m') = t'] = \frac{1}{\abs{\mathcal{T}}^2}.
\]
\end{proof}
\end{frame}
\begin{frame}\frametitle{Security of Construction from An SUF}
\begin{theorem}
If $h$ is an SUF, Construction is a $1/\abs{\mathcal{T}}-$secure MAC.
\end{theorem}
\begin{proof}
We assume that $\mathcal{A}$ is deterministic without loss of generality. The adversary receives tag $t’$ for the message $m’$, where $m’$ is fixed.
\[
\begin{split}
\Pr [\mathsf{Macforge}_{\mathcal{A},\Pi}^{1-\mathsf{time}} = 1] &=  \sum_{t' \in \mathcal{T}} \Pr [\mathsf{Macforge}_{\mathcal{A},\Pi}^{1-\mathsf{time}} = 1 \land h_k(m')=t'] \\
&= \sum_{t' \in \mathcal{T}} \Pr [h_k(m)=t \land h_k(m')=t'] \\
&= \sum_{t' \in \mathcal{T}} \frac{1}{\abs{\mathcal{T}^2}} =  \frac{1}{\abs{\mathcal{T}}}
\end{split}
\]
\end{proof}
\end{frame}
\begin{frame}\frametitle{Limitations on Information-Theoretic MACs}
Any $\ell$-time $2^{-n}$-secure MAC requires keys of length at least $(\ell +1) \cdot n$. 
\begin{theorem}
Let $\Pi$ be a 1-time $2^{-n}$-secure MAC where all keys are the same length. Then the keys must have length at least $2n$.
\end{theorem}
\begin{proof}
Let $\mathcal{K}(t) \overset{\mathsf{def}}{=} \{ k | \mathsf{Vrfy}_k(m, t) = 1\}$. For any $t$, $\abs{\mathcal{K}(t)} \leq 2^{-n} \cdot \abs{\mathcal{K}}$. Otherwise, $(m, t)$ would be a valid forgery with probability at least $\abs{\mathcal{K}(t)}/\abs{\mathcal{K}}> 2^{-n}$. The probability that $\mathcal{A}$ outputs a valid forgery is at least
\[ \sum_{t} \Pr [\mathsf{Mac}_k(m) = t] \cdot \frac{1}{\abs{\mathcal{K}(t)}} \geq \sum_{t} \Pr [\mathsf{Mac}_k(m) = t] \cdot \frac{2^n}{\abs{\mathcal{K}}} = \frac{2^n}{\abs{\mathcal{K}}} 
\]
As the probability is at most $2^{-n}$, $\abs{\mathcal{K}} \geq 2^{2n}$. Since all keys have the same length, each key must have length at least $2n$.
\end{proof} 
\end{frame}
\begin{frame}\frametitle{Summary}
\begin{itemize}
\item adaptive CMA, replay attack, birthday attack 
\item existential unforgeability, collision resistance
\item CBC-MAC, CRHF, Merkle-Damg\r{a}rd transform, NMAC, HMAC
\item information-theoretic MAC
\end{itemize}
\end{frame}
\end{document}
