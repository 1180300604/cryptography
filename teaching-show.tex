% presentation
\documentclass{beamer}
\usetheme[height=7mm]{Rochester}
\usecolortheme{rose}

% handout

%\documentclass[handout]{beamer}
%\usepackage{pgfpages} \pgfpagesuselayout{8 on 1}[a4paper]

%\documentclass[mathserif]{article}
%\usepackage{beamerarticle}

\usepackage{amsmath}
\usepackage{comment}
\usepackage{amssymb,amsfonts}
\usepackage[T1]{fontenc}
\usepackage{lmodern}
\usepackage{tikz}
\usepackage{simpsons}
\usepackage{marvosym}
\usepackage{color}
\usepackage{multirow}
\usepackage{pgffor}
\usepackage[slide,algoruled,titlenumbered,vlined,noend,linesnumbered,]{algorithm2e}

\usefonttheme{structurebold}

\setbeamertemplate{footline}[frame number]
\setbeamertemplate{navigation symbols}{}
\setbeamerfont{smallverb}{size*={73}}
\usefonttheme[onlymath]{serif}
\setbeamertemplate{theorems}[numbered]
\newtheorem{construction}[theorem]{Construction}
\newtheorem{proposition}[theorem]{Proposition}

\AtBeginSection[] {
  \begin{frame}
    \frametitle{Content}
    \tableofcontents[currentsection]
  \end{frame}
  \addtocounter{framenumber}{-1}
}

\usetikzlibrary[shapes.arrows]
\usetikzlibrary{shapes.geometric}
\usetikzlibrary{backgrounds}
\usetikzlibrary{positioning}
\usetikzlibrary{calc}
\usetikzlibrary{intersections}
\usetikzlibrary{fadings}
\usetikzlibrary{decorations.footprints}
\usetikzlibrary{patterns}
\usetikzlibrary{shapes.callouts}
\usetikzlibrary{fit}
%handout

\providecommand{\abs}[1]{\lvert#1\rvert}

\tikzset{every picture/.style={line width=1pt,show background rectangle},background rectangle/.style={fill=blue!10,rounded corners=2ex}}

\author{Yu Zhang}
\institute{Harbin Institute of Technology}
\date[Crypt'20A]{Cryptography, Autumn, 2020}

%% presentation
\documentclass{beamer}
\usetheme[height=7mm]{Rochester}
\usecolortheme{rose}

% handout

%\documentclass[handout]{beamer}
%\usepackage{pgfpages} \pgfpagesuselayout{8 on 1}[a4paper]

%\documentclass[mathserif]{article}
%\usepackage{beamerarticle}

\usepackage{amsmath}
\usepackage{comment}
\usepackage{amssymb,amsfonts}
\usepackage[T1]{fontenc}
\usepackage{lmodern}
\usepackage{tikz}
\usepackage{simpsons}
\usepackage{marvosym}
\usepackage{color}
\usepackage{multirow}
\usepackage{pgffor}
\usepackage[slide,algoruled,titlenumbered,vlined,noend,linesnumbered,]{algorithm2e}

\usefonttheme{structurebold}

\setbeamertemplate{footline}[frame number]
\setbeamertemplate{navigation symbols}{}
\setbeamerfont{smallverb}{size*={73}}
\usefonttheme[onlymath]{serif}
\setbeamertemplate{theorems}[numbered]
\newtheorem{construction}[theorem]{Construction}
\newtheorem{proposition}[theorem]{Proposition}

\AtBeginSection[] {
  \begin{frame}
    \frametitle{Content}
    \tableofcontents[currentsection]
  \end{frame}
  \addtocounter{framenumber}{-1}
}

\usetikzlibrary[shapes.arrows]
\usetikzlibrary{shapes.geometric}
\usetikzlibrary{backgrounds}
\usetikzlibrary{positioning}
\usetikzlibrary{calc}
\usetikzlibrary{intersections}
\usetikzlibrary{fadings}
\usetikzlibrary{decorations.footprints}
\usetikzlibrary{patterns}
\usetikzlibrary{shapes.callouts}
\usetikzlibrary{fit}
%handout

\providecommand{\abs}[1]{\lvert#1\rvert}

\tikzset{every picture/.style={line width=1pt,show background rectangle},background rectangle/.style={fill=blue!10,rounded corners=2ex}}

\author{Yu Zhang}
\institute{Harbin Institute of Technology}
\date[Crypt'20A]{Cryptography, Autumn, 2020}

%\input{1introduction.tex}
%\input{2perfectlysecret.tex}
%\input{3privatekey.tex}


\title{Introduction}

\begin{document}
\maketitle
\begin{frame}
\frametitle{Outline}
\tableofcontents
\end{frame}
\section{Cryptography and Modern Cryptography}
\begin{frame}\frametitle{What is Cryptography?}
\begin{itemize}
\item \textbf{Cryptography}: from Greek \emph{krypt\'os}, ``hidden, secret''; and \emph{gr\'{a}phin}, ``writing''
\item \textbf{Cryptography}: the art of writing or solving codes.\\ (Concise oxford dictionary 2006)
\item \textbf{Codes}: a system of prearranged signals, especially used to ensure secrecy in transmitting messages. \\ (\emph{code word} in cryptography)
\item \textbf{1980s}: from Classic to Modern; from Military to Everyone
\item \textbf{Modern cryptography}: the scientific study of mathematical techniques for securing digital information, systems, and distributed computations against adversarial attacks
\end{itemize}
\end{frame}
\section{The Setting of Private-Key Encryption}
\begin{frame}\frametitle{Private-Key Encryption}
\begin{itemize}
\item \textbf{Goal}: to construct \textbf{ciphers} (encryption schemes) for providing secret communication between two parties sharing \textbf{private-key} (the symmetric-key) in advance
\item \textbf{Implicit assumption}: there is some way of initially sharing a key in a secret manner
\item \textbf{Disk encryption}: the same user at different points in time
\end{itemize}
\end{frame}
\begin{frame}\frametitle{The Syntax of Encryption}
\begin{figure}
\begin{center}
\input{tikz/private-key}
\end{center}
\end{figure}
\begin{itemize}
\item key $k \in \mathcal{K}$, plaintext (or message) $m \in \mathcal{M}$, ciphertext $c \in \mathcal{C}$
\item \textbf{Key-generation} algorithm~$k \gets \mathsf{Gen}$
\item \textbf{Encryption} algorithm~$c:= \mathsf{Enc}_k(m)$
\item \textbf{Decryption} algorithm~$m:= \mathsf{Dec}_k(c)$
\item \textbf{Encryption scheme}: $\Pi = (\mathsf{Gen}, \mathsf{Enc}, \mathsf{Dec})$
\item \textbf{Basic correctness requirement}: $\mathsf{Dec}_k(\mathsf{Enc}_k(m)) = m$
\end{itemize}
\end{frame}
\begin{frame}\frametitle{Securing Key vs Obscuring Algorithm}
\begin{itemize}
\item Easier to maintain secrecy of a short key
\item In case the key is exposed, easier for the honest parties to change the key
\item In case many pairs of people, easier to use the same algorithm, but different keys
\end{itemize}
\begin{alertblock}{Kerckhoffs's principle}
\begin{quote}
The cipher method must not be required to be secret, and it must be able to fall into the hands of the enemy without inconvenience.
\end{quote}	
\end{alertblock}
\end{frame}
\begin{frame}\frametitle{Why ``Open Cryptographic Design''}
\begin{itemize}
\item Published designs undergo public scrutiny are to be stronger
\item Better for security flaws to be revealed by ``ethical hackers''
\item Reverse engineering of the code (or leakage by industrial espionage) poses a serious threat to security
\item Enable the establishment of standards.
\end{itemize}
\end{frame}
\begin{frame}\frametitle{Attack Scenarios}	
\begin{itemize}
\item \textbf{Ciphertext-only}: the adversary just observes ciphertext
\item \textbf{Known-plaintext}: the adversary learns pairs of plaintexts/ciphertexts under the same key
\item \textbf{Chosen-plaintext}: the adversary has the ability to obtain the encryption of plaintexts of its choice
\item \textbf{Chosen-ciphertext}: the adversary has the ability to obtain the decryption of \textbf{other} ciphertexts of its choice
\item \textbf{Passive attack}: COA KPA
\begin{itemize}
\item because not all ciphertext are confidential
\end{itemize}
\item \textbf{Active attack}: CPA CCA
\begin{itemize}
\item when to encrypt/decrypt whatever an adversary wishes?
\end{itemize}
\end{itemize}	
\end{frame}
\section{Historical Ciphers and Their Cryptanalysis}
\begin{comment}
	\begin{frame}\frametitle{Why We Learn Broken Ciphers?}
	\begin{itemize}
	\item To understand the weaknesses of an ``ad-hoc'' approach
	\item To learn that ``simple'' approaches are unlikely to succeed
	\item To feel that ``we are smart enough to do some crypt-analyzing''
	\end{itemize}
	\end{frame}
\end{comment}

\begin{frame}[fragile]\frametitle{Caesar's Cipher}
\begin{quote}
If he had anything confidential to say, he wrote it in cipher, that is, by so changing the order of the letters of the alphabet, that not a word could be made out. If anyone wishes to \alert{decipher} these, and get at their meaning, he must \alert{substitute the fourth letter of the alphabet, namely D, for A}, and so with the others

\rightline{--Suetonius,``Life of Julius Caesar''}
\end{quote}
\begin{itemize}
	\item $\mathsf{Enc}(m)=m+3\mod 26$ \footnote{In fact the quote indicates that decryption involved rotating letters of the alphabet forward 3 positions, $\mathsf{Dec}(c)=c+3\mod 26$}
	\item \textbf{Weakness}: ? %\alert{What is the key?}
\end{itemize}
\begin{exampleblock}{Example}
\verb|begintheattacknow|
%\verb|EHJLQWKHDWWDFNQRZ|
\end{exampleblock}
\end{frame}
\begin{frame}[fragile]\frametitle{Shift Cipher}
\begin{itemize}
\item $\mathsf{Enc}_k(m)=m+k\mod 26$
\item $\mathsf{Dec}_k(c)=c-k\mod 26$
\item \textbf{Weakness}: ? %Fragile under \textbf{Brute-force attack} (exhaustive search)
\end{itemize}
\begin{exampleblock}{Example: Decipher the string}	
\verb|EHJLQWKHDWWDFNQRZ|
\end{exampleblock}
\begin{alertblock}{Sufficient Key Space Principle}
Any secure encryption scheme must have a key space that is not vulnerable to exhaustive search.\footnote{If the plaintext space is larger than the key space.}
\end{alertblock}
\end{frame}
\begin{frame}\frametitle{Index of Coincidence (IC) Method (to find $k$)}
\textbf{Index of Coincidence (IC)}: the probability that two randomly selected letters (pick-then-return) will be identical.

Let $p_i$ denote the probability of $i$th letter in English text.
\[I \overset{\text{def}}{=}\sum_{i=0}^{25} p_i^2 \]
\begin{exampleblock}{Example}
What's the IC of `apple'?
\end{exampleblock}

For a long English text, the IC is $\approx 0.065$.
For $j = 0, 1, \dotsc , 25$, $q_j$ is the probability of $j$th letter in the ciphertext.
\[I_j \overset{\text{def}}{=}\sum_{i=0}^{25} p_i \cdot q_{i+j}\]
\alert{Q: For shift cipher, if $j = k$, then $I_j \approx$ ?}
\end{frame}

\begin{frame}[fragile]\frametitle{Mono-Alphabetic Substitution}
\begin{itemize}
\item \textbf{Idea}: To map each character to a different one in an arbitrary manner
\item \textbf{Strength}: Key space is large $\approx 2^{88}$. \alert{Q: how to count?}
\item \textbf{Weakness}: ? %The mapping of each letter is fixed
\end{itemize}
\begin{exampleblock}{Example}
\verb|abcdefghijklmnopqrstuvwxyz|\\
\verb|XEUADNBKVMROCQFSYHWGLZIJPT|

Plaintext: \verb|tellhimaboutme|\\
Ciphertext: \verb|??????????????|
\end{exampleblock}
\end{frame}
\begin{frame}[fragile]\frametitle{Attack with Statistical Patterns}
\begin{enumerate}
\item Tabulate the frequency of letters in the ciphertext
\item Compare it to those in English text
\item Guess the most frequent letter corresponds to \verb|e|, and so on
\item Choose the plaintext that does ``make sense'' (Not trivial)
\end{enumerate}
\begin{table}
\begin{center}
\caption{Average letter frequencies for English-language text}
\begin{tabular}{|cc|cc|cc|cc|cc|} \hline
e & 12.7\% & t & 9.1\% & a & 8.2\% & o & 7.5\% & i & 7.0\%\\
n & 6.7\% & \_ & 6.4\% & s & 6.3\% & h & 6.1\% & r & 6.0\%\\
d & 4.3\% & l & 4.0\% & c & 2.8\% & u & 2.8\% & m & 2.4\%\\
w & 2.4\% & f & 2.2\% & g & 2.0\% & y & 2.0\% & p & 1.9\%\\
b & 1.5\% & v & 1.0\% & k & 0.8\% & j & 0.2\% & x & 0.2\%\\
q & 0.1\% & z & 0.1\% & & & & & &\\ \hline
\end{tabular}
\end{center}
\end{table}
\end{frame}
\begin{frame}[fragile]\frametitle{Example of Frequency Analysis (Ciphertext)}
\begin{verbatim}
LIVITCSWPIYVEWHEVSRIQMXLEYVEOIEWHRXEXIPFEMVEWHKVS
TYLXZIXLIKIIXPIJVSZEYPERRGERIMWQLMGLMXQERIWGPSRIH
MXQEREKIETXMJTPRGEVEKEITREWHEXXLEXXMZITWAWSQWXSWE
XTVEPMRXRSJGSTVRIEYVIEXCVMUIMWERGMIWXMJMGCSMWXSJO
MIQXLIVIQIVIXQSVSTWHKPEGARCSXRWIEVSWIIBXVIZMXFSJX
LIKEGAEWHEPSWYSWIWIEVXLISXLIVXLIRGEPIRQIVIIBGIIHM
WYPFLEVHEWHYPSRRFQMXLEPPXLIECCIEVEWGISJKTVWMRLIHY
SPHXLIQIMYLXSJXLIMWRIGXQEROIVFVIZEVAEKPIEWHXEAMWY
EPPXLMWYRMWXSGSWRMHIVEXMSWMGSTPHLEVHPFKPEZINTCMXI
VJSVLMRSCMWMSWVIRCIGXMWYMX
\end{verbatim}
\end{frame}
\begin{frame}[fragile]\frametitle{Example of Frequency Analysis (Analysis)}
Count and Guess, Trial and Error.
\begin{table}
\begin{center}
\caption{Analysis Steps}
\begin{tabular}{|r|l|} \hline
Ciphertext & Plaintext \\ \hline
\alert{I}   & \alert{e} \\
\alert{XLI} & \alert{the} \\
\alert{E} & \alert{a} \\
\alert{R}tate & \alert{s}tate \\
atthatt\alert{MZ}e & atthatt\alert{im}e \\
he\alert{V}e & he\alert{r}e \\
remar\alert{A} & remar\alert{k} \\ \hline
\end{tabular}
\end{center}
\end{table}
\end{frame}
\begin{frame}[fragile]\frametitle{Example of Frequency Analysis (Plaintext)}
\begin{quote}
Hereupon Legrand arose, with a grave and stately air, and brought me the beetle
from a glass case in which it was enclosed. It was a beautiful scarabaeus, and, at
that time, unknown to naturalists -- of course a great prize in a scientific point
of view. There were two round black spots near one extremity of the back, and a
long one near the other. The scales were exceedingly hard and glossy, with all the
appearance of burnished gold. The weight of the insect was very remarkable, and,
taking all things into consideration, I could hardly blame Jupiter for his opinion
respecting it.

\rightline{--Edgar Allan Poe's ``The Gold-Bug''}
\end{quote}
\end{frame}

\begin{frame}[fragile]\frametitle{Vigen\`{e}re (poly-alphabetic shift) Cipher}
\begin{itemize}
\item \textbf{Idea}: To ``smooth out'' the distribution in the ciphertext by mapping different instances of the same letter in the plaintext to different ones in the ciphertext
\item \textbf{Encryption}: $c_i=m_i+k_{[i\bmod t]}$, $t$ is the length (period) of $k$
\item \textbf{Cryptanalysis}: Need find $t$; if $t$ is known, need know whether the decryption ``makes sense'', but brute force ($26^t$) is infeasible for $t > 15$
\end{itemize}
\begin{exampleblock}{Example (Key is `cafe')}
\begin{description}[Ciphertext]
\item[Plaintext]  \verb|tellhimaboutme| \\
\item[Key]        \verb|cafecafecafeca| \\
\item[Ciphertext] \verb|??????????????| %\verb|WFRQKJSFEPAYPF|
\end{description}
\end{exampleblock}
\end{frame}
\begin{frame}[fragile]\frametitle{Kasiski's Method (to find $t$)}
\begin{itemize}
\item To identify repeated patterns of length 2 or 3
\item The distance between such appearances is a multiple of $t$
\item $t$ is the greatest common divisor of all the distances
\end{itemize}
\begin{exampleblock}{Example (Key is `beads')}
\begin{semiverbatim}
themanandthewomanretrievedtheletterfromthepostoffice
beadsbeadsbeadsbeadsbeadsbeansdeadsbeadsbeadsbeadbea
VMFQTPFOH\alert{MJJ}XSFCSSIMTNFZXFYISEIYUIKHWPQ\alert{MJJ}QSLVTGJKGF
\end{semiverbatim}
\end{exampleblock}
\end{frame}
\begin{frame}\frametitle{Index of Coincidence (IC) Method (to find $t$)}
For $\tau = 1, 2, \dotsc$, $q_i$ is the probability of $i$th letter in $c_1, c_{1+\tau}, c_{1+2\tau}, \dotsc$, IC is
\[I_\tau \overset{\text{def}}{=}\sum_{i=0}^{25} q_i^2\]
\alert{If $\tau = t$, then $I_\tau \approx ?$} ; otherwise $q_i \approx \frac{1}{26}$ and
\[I_\tau \approx \sum_{i=0}^{25} \left(\frac{1}{26}\right)^2 \approx 0.038\]
Then reuse IC method to find $k_i$.
\begin{alertblock}{Arbitrary Adversary Principle}
Security must be guaranteed for any adversary within the class of adversaries having the specified power
\end{alertblock}
\end{frame}
\begin{frame}\frametitle{Cryptanalytic Attacks (homework assignment)}
\begin{itemize}
\item Under COA, the requirement for ciphertext related to the size of the key space.  Vig\`{e}nere > mono-alphabetic sub. > shift
\item Under KPA, trivially broken.
\end{itemize}
\begin{alertblock}{Lessons learned}
\begin{itemize}
\item Sufficient key space principle
\item Designing secure cipher is a hard task
\item Complexity does not imply security (then what does?)
\item Arbitrary adversary principle
\end{itemize}
\end{alertblock}
\end{frame}
\section{The Basic Principles of Modern Cryptography}
\begin{frame}\frametitle{Three Main Principles of Modern Cryptography}
\begin{enumerate}
\item The formulation of a rigorous \textbf{definition} of security / threat model
\item When the security of a cipher relies on an unproven \textbf{assumption}, this assumption must be precisely stated and be as minimal as possible
\item Cipher should be accompanied by a rigorous \textbf{proof} of security with the above definition and the above assumption
\end{enumerate}
\end{frame}
\begin{frame}\frametitle{Why Principle 1 -- Formulation of Exact Definitions}
\begin{exampleblock}{Q: how would you formalize the security for private-key encryption?}
\begin{enumerate}
\item \emph{No adversary can find the secret key when given a ciphertext.}\\
$\mathsf{Enc}_k(m)=m$
\item \emph{No adversary can find the plaintext that corresponds to the ciphertext.}\\
$\mathsf{Enc}_k(m)=m_{0}\| \mathsf{AES}_k(m)$
\item \emph{No adversary can determine any character of the plaintext that corresponds to the ciphertext.}\\
$m=1000$, someone can learn $ 800 < m < 1200$
\item \emph{No adversary can derive any meaningful information about the plaintext from the ciphertext.}\\
Could you define so-called `meaningful'?
\end{enumerate}
\emph{\alert{Definitions of security should suffice for all potential applications.}}
\end{exampleblock}
\end{frame}
\begin{frame}\frametitle{Why Principle 1 -- How to define}
%\begin{exampleblock}{General Form}
%A cryptographic scheme for a given \textbf{task} is secure if no adversary of a specified \textbf{power} can achieve a specified \textbf{break}
%\end{exampleblock}

How To Define Security -- Lesson From Alan Turing
\begin{itemize}
\item What's computation?\footnote{Q: Any ``mathematical proof that there exist well-defined problems that computers cannot solve''? A: Halting Problem in computability theory}
\begin{enumerate}
\item A direct appeal to \textbf{intuition}
\item A \textbf{proof of the equivalence} of two definitions\\ (The new one has a greater intuitive appeal)
\item Giving \textbf{examples} solved using a definition
\end{enumerate}
\item Additional method for security: \textbf{Test of time}
\end{itemize}
\end{frame}	
\begin{frame}\frametitle{Principle 2 -- Reliance on Precise Assumptions}
Most cryptographic constructions \textbf{cannot be proven secure unconditionally}
\begin{itemize}
	\item \textbf{Why?} 
	\begin{enumerate}
		\item Validation of the assumption
		\item Comparison of schemes
		\item Facilitation of proofs of security
	\end{enumerate}
	\textbf{The construction is secure if the assumption is true.}
	\item \textbf{How?} 
	\begin{enumerate}
		\item old, so well tested
		\item simple and lower-level, so easy to study, refute \& correct
	\end{enumerate}
\end{itemize}
\end{frame}
\begin{frame}\frametitle{Principle 3 -- Rigorous Proofs of Security}
\begin{itemize}
\item \textbf{Why?} Proofs are more desirable in computer security than in other fields.
\item \textbf{The reductionist approach}: 
\begin{theorem}	Given that Assumption X is true, Construction Y is secure according to the given definition.
\end{theorem}
\begin{proof} Reduce the problem given by X to the problem of breaking Y.
\end{proof}
\item \textbf{Ad-hoc approaches}: for those who need a ``quick and dirty'' solution, or who are just simply unaware.
\end{itemize}
\end{frame}
\begin{frame}\frametitle{Summary}
\begin{itemize}
\item Cryptography secures information, transactions and computations
\item Kerckhoffs's principle \& Open cryptographic design
\item Caesar's, shift, Mono-Alphabetic sub., Vigen\`{e}re
\item Brute force, letter frequency, Kasiski's, IC
\item Sufficient key space principle
\item Arbitrary adversary principle
\item Rigorously proven security
\end{itemize}
\end{frame}
\begin{frame}\frametitle{What is cryptography? [xkcd:504]}
\begin{figure}
\begin{center}
\includegraphics[width=100mm]{pic/legal} 
\end{center}
\end{figure}
\end{frame}
\begin{frame}\frametitle{Alice, Bob  [xkcd:1323]}
Changing the names would be easier, but if you're not comfortable lying, try only making friends with people named Alice, Bob, Carol, etc.
\begin{figure}
\begin{center}
\includegraphics[width=45mm]{pic/alice-bob} 
\end{center}
\end{figure}
\end{frame}
\end{document}


%\input{2perfectlysecret.tex}
%\input{3privatekey.tex}


\title{Chosen Plaintext Attack \\ and \\ Pseudorandom Function}

\begin{document}
\maketitle
\begin{frame}
\frametitle{Outline}
\tableofcontents
\end{frame}
\section{Chosen-Plaintext Attacks (CPA)}
\begin{frame}\frametitle{Electronic Code Book (ECB) Mode}
\begin{figure}
\begin{center}
\input{tikz/ECB}
\end{center}
\end{figure}
\begin{itemize}
\item \alert{Q: is it indistinguishable in the presence of an eavesdropper?}
\end{itemize}
\end{frame}
\begin{frame}\frametitle{Attack on ECB mode}
\begin{figure}
\begin{center}
\includegraphics[width=100mm]{pic/ecb} 
\end{center}
\end{figure}
\end{frame}
\begin{frame}\frametitle{Chosen-Plaintext Attacks (CPA)}
\textbf{CPA}: the adversary has the ability to obtain the encryption of plaintexts of its choice
\begin{exampleblock}{A story in WWII}
\begin{itemize}
\item Navy cryptanalysts believe the ciphertext ``AF'' means ``Midway island'' in Japanese messages
\item But the general did not believe that Midway island would be attacked
\item Navy cryptanalysts sent a plaintext that the freshwater supplies at Midway island were low
\item Japanese intercepted the plaintext and sent a ciphertext that ``AF'' was low in water
\item The US forces dispatched three aircraft carriers and won
\end{itemize}
\end{exampleblock}
\end{frame}
\begin{frame}\frametitle{Security Against CPA}
The CPA indistinguishability experiment $\mathsf{PrivK}^{\mathsf{cpa}}_{\mathcal{A},\Pi}(n)$:
\begin{enumerate}
	\item $k \gets \mathsf{Gen}(1^n)$
	\item $\mathcal{A}$ is given input $1^n$ and \textbf{oracle access} $\mathcal{A}^{\mathsf{Enc}_k(\cdot)}$ to $\mathsf{Enc}_k(\cdot)$, outputs $m_0, m_1$ of the same length
	\item $b \gets \{0,1\}$. Then $c \gets \mathsf{Enc}_k(m_b)$ is given to $\mathcal{A}$
	\item $\mathcal{A}$ \textbf{continues to have oracle access} to $\mathsf{Enc}_k(\cdot)$, outputs $b'$
	\item If $b' = b$, $\mathcal{A}$ succeeded $\mathsf{PrivK}^{\mathsf{cpa}}_{\mathcal{A},\Pi}=1$, otherwise 0
\end{enumerate}
\begin{figure}
\begin{center}
\input{tikz/pri-cpa-exp.tex}
\end{center}
\end{figure}
\end{frame}
\begin{frame}\frametitle{CPA Security}
\begin{definition}\label{def:cap-ind}
$\Pi$ has \textbf{indistinguishable encryptions under a CPA (CPA-secure)} if $\forall$ \textsc{ppt} $\mathcal{A}$, $\exists$ $\mathsf{negl}$ such that
\[ \Pr\left[\mathsf{PrivK}^{\mathsf{cpa}}_{\mathcal{A},\Pi}(n)=1\right] \le \frac{1}{2} + \mathsf{negl}(n).
\]
\end{definition}
\begin{itemize}
\item \alert{Q: Is any cipher we have learned so far CPA-secure? Why?}
\end{itemize}
\end{frame}
\section{Pseudorandom Functions}
\begin{frame}\frametitle{Concepts on Pseudorandom Functions}
\begin{figure}
\begin{center}
\input{tikz/keyed-func.tex}
\end{center}
\end{figure}
\begin{itemize}
\item \textbf{Keyed function} $F : \{0,1\}^* \times \{0,1\}^* \to \{0,1\}^*$ \\
$F_k : \{0,1\}^* \to \{0,1\}^*$, $F_k(x) \overset{\text{def}}{=} F(k,x)$
\item \textbf{Look-up table $f$}: $\{0,1\}^n \to \{0,1\}^n$ with size \alert{ = ? bits} %$n\cdot2^n$.
\item \textbf{Function family $\mathsf{Func}_n$}: all functions $\{0,1\}^n \to \{0,1\}^n$. $|\mathsf{Func}_n| = 2^{n\cdot2^n}$
\item \textbf{Length Preserving}: $\ell_{key}(n) = \ell_{in}(n) = \ell_{out}(n)$
\end{itemize}
\end{frame}
\begin{frame}\frametitle{Definition of Pseudorandom Function}
\textbf{Intuition}: A PRF $F$ generates a function $F_k$ that is indistinguishable from truly random selected function $f$ (look-up table) in $\mathsf{Func}_n$.\\ However, the function has \textbf{exponential length}. Give $D$ the deterministic \textbf{oracle access $D^{\mathcal{O}}$} to the functions $\mathcal{O}$.
\begin{definition}
An efficient length-preserving, keyed function $F$ is a \textbf{pseudorandom function (PRF)} if
$\forall\;$ \textsc{ppt} distinguishers $D$,
\[ \left|\Pr[D^{F_k(\cdot)}(1^n)=1] - \Pr[D^{f(\cdot)}(1^n)=1]\right| \le \mathsf{negl}(n),
\]
where $f$ is chosen \emph{u.a.r} from $\mathsf{Func}_n$.
\end{definition}
\begin{alertblock}{Q: Is the fixed-length OTP a PRF?}
\end{alertblock}
%\textbf{PRG vs. PRF}:
%\begin{itemize}
%\item Pseudorandomness over a set of strings vs. a set of functions.
%\item A PRG --- an instance of keyed PRF.
%\end{itemize}
%\textbf{Existence}: if PRG exists. In practice, block ciphers may be PRF.
\end{frame}
\begin{frame}\frametitle{Questions}
\begin{exampleblock}{Let $F: \{0,1\}^{n} \times \{0,1\}^{n} \to \{0,1\}^{n}$ be a secure PRF. Is $G$ a secure PRF?}
\begin{itemize}
\item $G((k_{1},k_{2}), x) = F(k_{1},x) \| F(k_{2},x)$
\item $ G(k,x) = \left\{ 
  \begin{array}{l l}
    F(k,x) & \quad \text{when}\ x \neq 0^{n}\\
    0^{n} & \quad \text{otherwise}\\
  \end{array} \right. $
\item $G(k,x) = F(k,x)\bigoplus F(k, x\oplus 1^{n})$
\end{itemize}
\end{exampleblock}
\end{frame}
\section{Constructing CPA-Secure Encryption Schemes}
\begin{frame}\frametitle{CPA-Security from Pseudorandom Function}
\begin{columns}[t]
\begin{column}{4cm}
\begin{figure}
\begin{center}
\input{tikz/encryptionwithpf}
\end{center}
\end{figure}
\end{column}
\begin{column}{6cm}
\begin{construction}\label{thm:cpa}
\begin{itemize}
\item Fresh random string $r$.
\item $F_k(r)$: $\abs{k} = \abs{m} = \abs{r} = n$.
\item $\mathsf{Gen}$: $k \in \{0,1\}^n$.
\item $\mathsf{Enc}$: $s := F_k(r)\oplus m$, $c := \left<r, s\right>$.
\item $\mathsf{Dec}$: $m := F_k(r)\oplus s$.
\end{itemize}
\end{construction}
\begin{theorem}\label{thm:prf}
If $F$ is a PRF, this fixed-length encryption scheme $\Pi$ is CPA-secure.
\end{theorem}
\end{column}
\end{columns}
\end{frame}
\begin{frame}\frametitle{Proof of CPA-Security from PRF}
\textbf{Idea}: First, analyze the security in an idealized world where $f$ is used in $\tilde{\Pi}$; next, claim that if $\Pi$ is insecure when $F_k$ was used then this would imply $F_k$ is not PRF by reduction.
\begin{proof}
(1) Analyze $\Pr[\mathsf{Break}]$, $\mathsf{Break}$ means $\mathsf{PrivK}_{\mathcal{A},\tilde{\Pi}}^{\mathsf{cpa}}(n) = 1$:  \\
$\mathcal{A}$ collects $\{ \left< r_i, f(r_i) \right> \}$, $i=1,\dots,q(n)$ with $q(n)$ queries; \\
The challenge $c=\left<r_c, f(r_c)\oplus m_b\right>$. \\
\begin{itemize}
\item $\mathsf{Repeat}$: $r_c \in \{ r_i \}$ with probability $\frac{q(n)}{2^n}$. $\mathcal{A}$ can know $m_b$.
\item $\overline{\mathsf{Repeat}}$: As OTP, $\Pr[\mathsf{Break}]=\frac{1}{2}$ 
\end{itemize}
\[
\begin{split}
	\Pr[\mathsf{Break}] & =\Pr[\mathsf{Break} \land \mathsf{Repeat}] + \Pr[\mathsf{Break} \land \overline{\mathsf{Repeat}}] \\
	&\le \Pr[\mathsf{Repeat}] + \Pr[\mathsf{Break} | \overline{\mathsf{Repeat}}] \\
	&\le \frac{q(n)}{2^n} + \frac{1}{2}.
\end{split}
\]
\end{proof}
\end{frame}
\begin{frame}\frametitle{Proof of CPA-Security from PRF (Cont.)}
\begin{proof}
(2) Reduce $D$ to $\mathcal{A}$:
\begin{figure}
\begin{center}
\input{tikz/pgfD}
\end{center}
\end{figure}
{\footnotesize 
$ \Pr[D^{F_k(\cdot)}(1^n)=1] = \Pr[\mathsf{PrivK}_{\mathcal{A},\Pi}^{\mathsf{cpa}}(n) = 1] = \frac{1}{2} + \varepsilon(n). $
$ \Pr[D^{f(\cdot)}(1^n)=1] = \Pr[\mathsf{PrivK}_{\mathcal{A},\tilde{\Pi}}^{\mathsf{cpa}}(n) = 1] = \Pr[\mathsf{Break}] \le \frac{1}{2} + \frac{q(n)}{2^n}. $
$\Pr[D^{F_k(\cdot)}(1^n)=1] - \Pr[D^{f(\cdot)}(1^n)=1] \ge \varepsilon(n) - \frac{q(n)}{2^n}.$
$\varepsilon(n)$ is negligible.
}
\end{proof}
\end{frame}

\begin{frame}\frametitle{Cipher Block Chaining (CBC) Mode}
\begin{figure}
\begin{center}
\input{tikz/CBC}
\end{center}
\end{figure}
\end{frame}
\begin{frame}\frametitle{Output Feedback (OFB) Mode}
\begin{figure}
\begin{center}
\input{tikz/OFB}
\end{center}
\end{figure}
\end{frame}
\begin{frame}\frametitle{Counter (CTR) Mode}
\begin{figure}
\begin{center}
\input{tikz/CTR}
\end{center}
\end{figure}
\begin{itemize}
\item $ctr$ is an $IV$
\end{itemize}
\end{frame}


\begin{frame}[fragile]\frametitle{$IV$ Should Not Be Predictable}
If $IV$ is predictable, then CBC/OFB/CTR mode is not CPA-secure.\\
\alert{Q: Why? (homework)}
\begin{exampleblock}{Bug in SSL/TLS 1.0}
$IV$ for record $\#i$ is last CT block of record $\#(i-1)$.
\end{exampleblock}
\begin{exampleblock}{API in OpenSSL}
\verb#void AES_cbc_encrypt (# \\
\verb#    const unsigned char *in,# \\
\verb#    unsigned char       *out,# \\
\verb#    size_t              length,# \\
\verb#    const AES_KEY       *key,# \\
\verb#    unsigned char       *ivec,   #  \alert{\textbf{User supplies $IV$}} \\
\verb#    AES_ENCRYPT or AES_DECRYPT);# \\
\end{exampleblock}
\end{frame}
\end{document}
